\chapter{Implementasi}


\section{Sisi Pendengar}
Pada implementasi fitur penyeimbang muat dari sisi pendengar, HAPHAProxy menyediakan satu alamat server yang bisa diakses oleh pendengar dan meneruskannya ke tiga node akhir. Untuk menyokong sistem penyeimbang muat, disediakan tiga Node akhir pada alamat masing-masing \texttt{10.151.36.201}, \texttt{10.151.36.202}, dan \texttt{10.151.36.203}. Ketiga Node akhir ini sudah terpasang server Icecast dengan konfigurasi bawaan. Konfigurasi \texttt{haproxy.cfg} dapat dilihat pada Listing \ref{lst:haproxycfg} berikut.



%\label{lst:haproxycfg}
%\caption{Konfigurasi HAProxy pada penyeimbang muat}
\begin{lstlisting}[breaklines,frame=single]
frontend localnodes
bind *:8000
mode http
default_backend nodes

backend nodes
mode http
balance leastconn
option forwardfor
option httpchk GET /
server ice1 10.151.36.201:8000 check maxconn 100000
server ice2 10.151.36.202:8000 check maxconn 100000
server ice3 10.151.36.203:8000 check maxconn 100000

\end{lstlisting}

Konfigurasi HAProxy di atas berjalan di server dengan alamat 10.151.36.205. Fitur penyeimbang muat ini bekerja pada port 8000 dan akan meneruskan permintaan yang masuk ke tiga node akhir yang sudah dituliskan di dalam konfigurasi. 
\section{Sisi Penyiar/Pengirim}

\subsection{Basis Data}
Dalam penelitian ini digunakan basis data berbasis dokumen MongoDB (\url{http://mongodb.org}) . Secara umum struktur dari basis data dalam penilitian ini terdiri dari tiga koleksi yaitu :

\begin{enumerate}
    \item \textbf{Peers} 
    
    Koleksi ini berisi daftar Node akhir yang bisa digunakan untuk melayani pendengar. Kolom yang digunakan untuk menyimpan informasi node akhir adalah sebagai berikut :
    
    \begin{itemize}
    \item Kolom \texttt{ip} merupakan alamat dari node akhir yang tersedia
    \item Kolom \texttt{online} untuk menginformasikan apakah node dengan alamat yang dimaksud dapat digunakan atau tidak.
    \end{itemize}
    
    \item \textbf{Users}
    Koleksi ini digunakan untuk menyimpan daftar pengguna yang berhak untuk melakukan pengiriman data siar pada sistem yang dibangun. Terdapat dua kolom yaitu kolom \texttt{userName} dan kolom \texttt{password}. Untuk penyimpanan di kolom \texttt{password} digunakan mekanisme \emph{hashing} berbasis \textbf{bcrypt} untuk menjaga keamanan autentikasi pengirim.
    
    \item \textbf{Streams}
    
    Setiap pengirim yang memiliki akses penyiaran ke server harus menentukan sendiri \textit{mount point} yang akan digunakan di dalam server. Koleksi ini akan mendaftar semua \textit{mount point} pilihan pengirim yang nantinya akan diinfokan ke pendengar \textit{mount point} mana yang sedang aktif. 
    
    Yang disimpan di dalam koleksi ini adalah nama \texttt{mountPoint} untuk alamat \textit{stream} dan pengirim yang bisa menggunakan \textit{stream} tersebut.
\end{enumerate}

\subsection{Penyebar data audio}

Untuk menyebarkan data audio dari aplikasi pengirim ke banyak Node akhir, dilakukan proses implementasi program berbasis \texttt{socket} berbasis Node.js. Program ini berfungsi untuk melakukan penyaluran atau \emph{piping} data aliran yang diterima ke banyak Node akhir yang didefinisikan pada koleksi \textbf{Peers} pada basis data. 

Cara kerja dari penyebar adalah sebagai berikut :
\begin{enumerate}
\item Ketika menerima koneksi dari klien penyiar, lakukan proses pengecekan data autentikasi (melalui header Authorization) untuk kemudian dicocokkan dengan basis data pada koleksi \textbf{Users}.
\item Jika autentikasi valid, dilakukan pengecekan \emph{mount point} yang digunakan oleh penyiar pada koleksi \textbf{Streams} apakah dimiliki oleh akun pengguna bersangkutan.
\item Kemudian jika sesuai dengan basis data, aliran data dari klien dialihkan ke semua Node akhir yang tercatat pada koleksi \textbf{Peers} pada basis data secara sekaligus.
\end{enumerate}

Program ini dibuat sehingga setransparan mungkin sehingga bisa diakses langsung melalui aplikasi multimedia yang mendukung Icecast seperti Mixxx atau VLC. 
