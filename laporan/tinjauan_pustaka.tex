\chapter{Tinjauan Pustaka}
\section{Teori Penunjang}
\begin{enumerate}
    \item HAProxy
    
    HAProxy ( High Availability Proxy ) merupakan salah satu sistem penyeimbang muat handal pada protokol TCP/HTTP. Sistem yang digunakannya akan membagi beban kerja ke sekumpulan server untuk memaksimalkan kerjanya.
    
    Beberapa alasan mengapa HAProxy ini banyak digunakan adalah :
    \begin{enumerate}
        \item Sangat cepat.
        \item Efisien, dengan 700 permintaan dalam satu detik CPU yang digunakan kurang dari 5\% dan 40MB RAM.
        \item Memungkinkan untuk dilakukan perubahan konfigurasi selama koneksi masih terjadi dan tidak akan mengganggu koneksi tersebut.
        \item Memungkinkan adanya \textit{queue} dalam pengaplikasiannya jika memang koneksi yang masuk terlalu banyak.
    \end{enumerate}
    
    
    \item Icecast
    
    Merupakan kumpulan dari program dan pustaka yang digunakan untuk melakukan \textit{audio streaming} melalui internet. Ada tiga komponen utama di dalamnya yaitu :
    
    \begin{enumerate}
        \item Icecast
        
        Program ini akan menyampaikan data stream audio ke pendengar.
        
        \item Libshout
        
        Adalah sebuah pustaka yang digunakan untuk mengirimkan data ke server. Data ini berasal dari orang yang ingin menyebarkan audionya sehingga bisa didengarkan orang lain.
        
        \item IceS
        
        Program yang akan mengirimkan data audio dari ke server dan disebarkan secara broadcast ke pendengar. Program ini dapat mengirim data audio di dalam media penyimpanan seperti berkas Ogg Vorbis atau suara langsung yang ditangkap sound card.
    \end{enumerate}
    
    \item NodeJS
    
    Merupakan sebuah platform yang dibangun di atas Chrome's JavaScript runtime dengan teknologi V8 yang mendukung proses server yang bersifat \textit{long-running}. Tidak seperti platform modern yang mengandalkan multithreading, NodeJS memilih menggunakan asynchronous I/O eventing. Karena inilah NodeJS mampu bekerja dengan konsumsi memori rendah.
    
    
    
\end{enumerate}

\section{Solusi Sebelumnya}

Beberapa pengembangan terhadap terknologi di atas antara lain:

\begin{enumerate}
    \item GitHub
    
    Sebagai media untuk melakukan \textit{versioning} dalam pengembangan perangkat lunak, pengguna mengharapkan kecepatan akses terhadap halaman pribadinya. Oleh karena itu pengembang Github memanfaatkan fitur \textit{proxy} pada HAProxy untuk membagi kinerja server di dalamnya sehingga pengguna dapat dengan cepat mengakses halaman pribadinya. (mojombo, 2009)
    
    \item Stack Overflow / Server Fault
    
    Pengembangnya jelas menyebutkan bahwa mereka adalah penggemar HAProxy yang telah berhasil me-\textit{load balancing}-kan 2 hingga 3 komputer server yang dimilikinya.
\end{enumerate}
